\subsection{The Buchsbaum-Rim multiplicity}
\label{multiplicities section}
In this section we recall some basic facts and the definitions of the multiplicities that we consider. For a general reference, see e.g.\ \autocite[Chapter 2]{roberts}.

Let $(A, \m)$ be a Noetherian local ring of dimension $n$. Let $I\subset \m$ be an $\m$-primary ideal. Then $A/I $ has finite length. Moreover, for $\ell \in \N$ large enough
\begin{equation*}
    \length (A/I^{\ell})
\end{equation*}
is a polynomial in $\ell$ of degree $n$. The \emph{Hilbert-Samuel multiplicity} $\ehs(I)$ is defined as the following normalisation of the leading term coefficient
\begin{equation*}
\ehs(I) :=  n! \operatorname{coeff} (\ell^n, \length(A/I^{\ell}))
\end{equation*}
where $\ell \gg 1$.
In fact, the multiplicity depends only on the \emph{integral closure} $\bar{I}$ of the ideal $I$. An element $x\in A$ is \emph{integral over $I$} precisely if there is a monic equation 
\begin{equation*}
    x^m + a_1x^{m-1}+ \cdots + a_m  = 0
\end{equation*}
with $a_k \in I^k$ and the ideal $\bar{I}$ consists precisely of all $x\in A$ that are integral over $I$. Note that $I \subset \bar{I}$. An ideal $J \subset I$ such that $I \subset \bar{J}$, i.e.\ such that all elements of $I$ are integral over $J$, is said to be a \emph{reduction} of the ideal $I$. If $I, J$ are $\m$-primary ideals such that $J$ is a reduction of $I$, then $e(J ) = e(I)$.

Suppose that $N \subset \m F$ is a submodule of a free $A$-module $F$ of rank $r$ such that $M=F/N$ is of finite length. 
The symmetric algebra $S(F)$ can be identified with the polynomial ring $A[X_1, \dots, X_r]$ as follows. Fix a basis $f_1, \dots, f_r$ of $A$. Let $\phi: S(F) \to A[X_1, \dots, X_r]$ be the homomorphism $\phi(f_k) := X_k$. The \emph{Rees ring} $R(N)$ of $N$ is the subring generated by $\phi(N)  \subset A[X_1,\dots, X_r]$. Let $S_{\ell}(F)$ and $R_{\ell}(N)$ denote the submodules of $S(F)$ and $R(N)$, respectively, containing homogenoeus polynomials of degree $\ell$.
For large enough $\ell \in \N$
\begin{equation*}
    \length(S_{\ell}(F)/R_{\ell}(N))
\end{equation*}
is a polynomial in $\ell$ of degree $n+r-1$.
The Buchsbaum-Rim multiplicity $\ebr(M)$ is then defined as
\begin{equation}
    \label{brmultdef}
    \ebr(M):= (n+r-1)! \operatorname{coeff}(\ell^{n+r-1}, \length(S_{\ell}(F)/R_{\ell}(N))
\end{equation}
where $\ell \gg 1$.

Let $I_1, \dots, I_r \subset \m$ be $\m$-primary ideals. For any $\ell=(\ell_1, \dots, \ell_r) \in \N^r$, it holds that
\begin{equation*}
    e(I_1^{\ell_1} \cdots I_r^{\ell_r})
\end{equation*}
is a homogeneous polynomial of degree $n$ in $\ell_1, \dots, \ell_r$.
Let $\alpha = (\alpha_1, \dots, \alpha_r) \in \N^r$ be a multi-index with $|\alpha|=n$. The mixed multiplicity $e_{\alpha}(I_1, \dots, I_r)$ of type $\alpha$ of the ideals $I_1, \dots, I_r$ is defined as
\begin{equation}
    \label{mixedmultdef}
    \binom{n}{\alpha} \ehs_{\alpha} (I_1, \dots, I_r):= \operatorname{coeff}\left( \ell_1^{\alpha_1}\cdots \ell_r^{\alpha_r}, \ehs(I_1^{\ell_1} \cdots I_r^{\ell_r}) \right).
\end{equation} 
In fact, we can calculate the mixed multiplicity as the Hilbert-Samuel multiplicity of an ideal by the following proposition (see e.g.\ \autocite[Lemma 2.5]{Swanson}).
\begin{prop}
        \label{mixed mult as hs mult}
    Let $I_1, \dots, I_r \subset A$ be $\m$-primary and $\alpha\in \N^r$ a multi-index with $|\alpha|=n$. Let $J$ be the ideal generated by $\alpha_1$ generic elements of $I_1$, $\alpha_2$ generic elements of $I_2$, \dots, $\alpha_r$ generic elements of $I_r$. Then 
    \begin{equation}
        e_{\alpha}(I_1, \dots, I_r) = e(J).
    \end{equation}
\end{prop}
Note that, as a consequence, for any $\m$-primary ideal $I$ it holds that
\begin{equation}
\label{mixed mult on diagonals}
    e_{\alpha}(I, \dots, I) = e(I),
\end{equation}
for any $\alpha\in \N^r$. This is because the ideal $J$ constructed in \cref{mixed mult as hs mult} is a reduction of $I$.

\begin{lem}[Kirby-Rees, 
 \autocite{KirbyRees2}]
Let $(A, \m) $ be a local Noetherian ring of dimension $n$. Let $I_1, \dots, I_r$ be $\m$-primary ideals and let $M = \bigoplus_{k=1}^r A/I_k$, so that $M$ is an $A$-module of finite length. Then the Buchsbaum-Rim multiplicity is given by 
\begin{equation}
    \ebr(M) = \sum_{|\alpha|=n} \ehs_{\alpha} (I_1, \dots, I_r),
\end{equation}
where $\ehs_{\alpha}$ is the mixed multiplicity of type $\alpha\in \N^r$.
\end{lem}

Let $X$ be a neighbourhood of the origin $0\in \C^n$ and consider a morphism $f: E \to Q$ of trivial holomorphic bundles over $X$. If $Z(f)=\{0\}$, where $Z(f)$ is the set where $f$ is not surjective, then $\M:=\O(Q)/\im f$ is an Artinian $\O_X$-module with support at the origin, i.e.\ $\M_z = 0$ if $z\neq 0$. It can thus be identified with the module $M:=\M_0$ which is a module of finite length over the local Noetherian ring $(\O_0, \m_0)$. We can therefore define the Buchsbaum-Rim multiplicity $\ebr(\M)$ of $\M$ as $\ebr(\M) = \ebr(M)$. Similarily, when $Q$ is the trivial line bundle, so that $f = (f_1, \dots, f_m)$, we can define the Hilbert-Samuel multiplicity of the Artinian ideal $\I$ that $f$ defines, as $\ehs(\I) = \ehs(I)$, where $I = \I_{0}$. The main result we need from this section is that if $f$ is block diagonal as in \cref{ThmB}, so that $\M \cong \oplus_{k=1}^r\O_X/\I_k$, then 
\begin{equation}
\label{brmixrelation}
 \ebr(\M) = \sum_{|\alpha|=n} \ehs_{\alpha} (\I_1, \dots, \I_r).
\end{equation}


% \begin{lem}[{[Referens!!]}]
% Let $(A, \m) $ be a local Noetherian ring of dimension $n$. Suppose $I_1, \dots, I_r$ have cofinite length. Then
% \begin{equation}
%     \ehs(I_1\cdots I_r) = \sum_{|\alpha|=n} \binom{n}{\alpha} \ehs_{\alpha} (I_1, \dots, I_r)
% \end{equation}
% where $\ehs_{\alpha}$ is the mixed multiplicity of type $\alpha$.
% \end{lem}
