\subsubsection*{Proof of Proposition 2.3}
Throughout this section, we let $$\A = \{ \text{Artinian ideals $J \subset \O_X$ supported at the origin}\} \cup \{\O_X\}.$$ Then $\A$ is a commutative monoid with the product defined by multiplying ideals.
For $J_1, \dots, J_n \in \A$ we define $m(J_1, \dots, J_n)$ as the number
\begin{equation}
    m(J_1, \dots, J_n) = \int\limits_{\{0\}} dd^c\log|g_1|^2 \wedge \cdots \wedge dd^c\log|g_n|^2
\end{equation}
where $g_k=(g_{k1}, \dots, g_{km_k})$ are tuples generating $J_k$. 
\begin{prop}
\label{properties of m}
    The function $m: \A^n \to \R$ is well-defined, symmetric, and multilinear.
\end{prop} 
\begin{proof}
To prove that $m$ is well-defined, we need to show that it is independent of the generators of the ideals.
Let $J_1= (u_1, \dots, u_p) = (v_1, \dots, v_q) \in \A$, and $J_k:=(g_{k1},\dots, g_{km_k}) \in \A$, for $k=2, \dots, n$. 
Since the Monge-Ampère product is symmetric (see \cref{MA section}) it suffices to show that 
\begin{multline}
\label{m well-defined}
  \int_{\{0\}} dd^c\log|u|^2 \wedge dd^c\log|g_2|^2 \wedge \cdots \wedge dd^c\log|g_n|^2   =\\ \int_{\{0\}} dd^c \log|v|^2 \wedge dd^c\log|g_2|^2 \wedge \cdots \wedge dd^c\log|g_n|^2.
\end{multline}

Now, note that $u = v A$ for some holomorphic matrix $A$ with positive rank on $X$. Hence, we have 
    \begin{equation*}
        \log|u|^2 \leqslant \log|v|^2 + \phi 
    \end{equation*}
    where $\phi = \log\|A\|_{op}^2$ is a locally bounded function. Similarily, since $v = uB$ for some $B$ of positive rank, we get
    \begin{equation*}
        \log |v|^2 \leqslant \log |u|^2 + \psi
    \end{equation*}
    for some locally bounded $\psi$. Thus, 
    \begin{equation*}
        \lim_{z\to 0} \frac{\log|u|^2}{\log|v|^2}  = 1
    \end{equation*}
    and thus, \eqref{m well-defined} follows from the first comparison theorem (see \autocite[Theorem III.7.1]{demailly}{}). Hence, $m$ is well-defined.

    That $m$ is symmetric follows immediately from the fact that the mixed Monge-Ampère product is symmetric.

    It remains to show that $m$ is multilinear. 
    Since $m$ is symmetric, it is enough to show 
    \begin{equation}
    \label{m multilinear}
        m(IJ, J_2, \dots, J_n) =  m(I, J_2, \dots, J_n) + m(J, J_2, \dots, J_n).
    \end{equation}
    Suppose $I = (u_1, \dots, u_p)$, $J=(v_1, \dots, v_q) \in\A$.
    Then $IJ$ is generated by $h_{k\ell} := u_kv_{\ell}$, for $k=1, \dots, p$ and $\ell=1, \dots, q$. Let $h = (h_{11},\dots, h_{1q}, \dots, h_{p1}, \dots, h_{pq})$.
    Then clearly $|h|^2 = |u|^2|v|^2$, and hence, $\log|h|^2 =  \log|u|^2+\log|v|^2$. Thus, \eqref{m multilinear}, follows from the multilinearity of the mixed Monge-Ampère product (see \cref{MA section}). This finishes the proof.
\end{proof}

Now, let $\gamma \in \N^n$ be the multi-index with $\gamma_k = 1 $ for $k= 1, \dots, n$. For $J_1, \dots, J_n \in \A$ we define $e(J_1, \dots, J_n)$ as the number
\begin{equation}
    e(J_1, \dots, J_n ) = e_{\gamma} (J_1, \dots, J_n).
\end{equation}

\begin{prop}
\label{properties of e}
    The function $e: \A^n \to \R$ is symmetric and multilinear.
\end{prop}
\begin{proof}
That $e$ is symmetric is clear in view of \eqref{mixedmultdef}.

Let $J_1, \dots, J_{n-1}, I, J \in \A$. Then by \autocite[Lemma 2.5]{Rees} we have linearity in the last factor
\begin{equation*}
    e(J_1, \dots, J_{n-1}, IJ) = e(J_1, \dots, J_{n-1}, I) + e(J_1, \dots, J_{n-1}, J).
\end{equation*}
Since $e$ is symmetric, it follows that it is multilinear.
\end{proof}
We want to show that $e = m$, and to do this we invoke the following.
\begin{prop}
\label{comparing symmetric forms}
Suppose $\psi_1, \psi_2 :\A^n \to \R$ are symmetric and multilinear such that for all $a\in \A$ we have
$\psi_1(a, \dots, a) = \psi_2(a,\dots, a)$. Then $\psi_1 = \psi_2$.
\end{prop}
This is immediate from the following elementary polarisation formula (written in multiplicative notation rather than the usual additive, since we are multiplying ideals).
\begin{prop}
\label{polarisation formula}
    Suppose $\A$ is a commutative monoid and  
    let $\psi: \A^n \to \R$ be symmetric and multilinear. Define $\Psi: \A\to \R$ by $\Psi(a) =  \psi(a, \dots, a) $.
    Then it holds that
    \begin{equation*}
        \psi(a_1, \dots, a_n)  = \frac{1}{n! }  \sum_{k=1}^n (-1)^k\sum_{1\leqslant i_1 < \cdots < i_k \leqslant n}  \Psi(a_{i_1}\cdots a_{i_k}). 
    \end{equation*}
\end{prop}

\begin{proof}[Proof of \cref{mma lelong numbers}]
 The functions $e, m : \A^n \to \R$ are multilinear and symmetric. From \eqref{mixed mult on diagonals} we have for any $I\in \A$ that
 $e(I, \dots, I) = e(I)$. From King's formula,
 \eqref{kings formel}, we get $m(I, \dots, I) = e(I)$, whence $m(I, \dots, I)  = e(I, \dots, I)$ follows.
 Thus, from \cref{comparing symmetric forms} we conclude that $e = m$.

Now, given ideals $I_1, \dots, I_r \in \A$ and a multi-index $\alpha\in \N^r$ as in the formulation of \cref{mma lelong numbers}, we define $J_1, \dots, J_n \in \A$ as follows. 
Let 
\begin{align*}
    J_k &= I_1, \quad \text{for $k=1, \dots, \alpha_1$} \\
    J_k &= I_2, \quad \text{for $k=\alpha_1+1, \dots, \alpha_2$} \\
    \vdots \\
     J_k &= I_r, \quad \text{for $k=\alpha_1+\cdots +\alpha_{r-1}+1, \dots, \alpha_1 +\cdots +\alpha_r$} .
\end{align*}
Let $\gamma \in \N^n $ be the multi-index with $\gamma_k = 1$ for $k=1, \dots, n$. It follows from \cref{mixed mult as hs mult} that 
\begin{equation*}
    e_{\gamma}(J_1, \dots, J_n) = e_{\alpha}(I_1, \dots, I_r).
\end{equation*}
As a consequence, $e(J_1, \dots, J_n) = e_{\alpha}(I_1, \dots, I_r)$, whence
\begin{multline}
   \int_{\{0\} }(dd^c \log|f|^2)^{\alpha} = m(J_1, \dots, J_n) = \\ e(J_1, \dots, J_n) = e_{\alpha} (I_1, \dots, I_r),
\end{multline}
which is precisely what we wanted to prove. 
\end{proof}