\begin{abstract}
ajsje fej
\end{abstract}
% Let $f$ be a holomorphic $(r\times m)$-matrix defined near the origin in $\C^n$ and with full rank outside the origin. 
% To the submodule $N$ of $\O^r_0$ defined by the image of $f$ there is a notion of multiplicity called the Buchsbaum-Rim multiplicity of $N$. 
% This is the leading coefficient of a Hilbert polynomial of a certain graded algebra defined from $N$. 
% In the special case when $f$ is a row matrix, the image is given by the ideal in $\O_0$ defined by
%  $f$ and the Buchsbaum-Rim multiplicity coincides with the classical Hilbert-Samuel multiplicity. 

% In this thesis we represent the Buchsbaum-Rim multiplicity $\ebr(N)$ of $N$ in terms of (residue) currents in the special case when the matrix $f$ is block diagonal. 
% More precisely, we prove that the point mass $\ebr(N) [0]$ factors into a product of a smooth form and a residue current associated to the so-called Buchsbaum-Rim complex of $f$. 
% This generalises a result in \autocite{AnderssonLelong}, where a similar factorisation is proven for row matrices and Hilbert-Samuel multiplicities. 
% When $f$ is block diagonal, the Buchsbaum-Rim multiplicity is given as a sum of so-called mixed multiplicities. 
% By King's formula, these multiplicities can be expressed with mixed Monge-Ampère products. 
% We show that the mixed Monge-Ampère products can be represented as the product of the smooth form and residue current defined from the Buchsbaum-Rim complex of $f$.

% \textbf{Keywords:} Buchsbaum-Rim multiplicity, mixed multiplicity, Buchsbaum-Rim complex, residue current, holomorphic morphism, Mongè-Ampere product, King's formula